%%%%%%%%%%%%%%%%%%%%%%%%%%%%%%%%%%%%%%%%%
% Twenty Seconds Resume/CV
% LaTeX Template
% Version 1.0 (8/1/25)
%
% This template has been downloaded from:
% https://github.com/ArthurGouing
%
% Original author:
% Carmine Spagnuolo (cspagnuolo@unisa.it) with major modifications by 
% Vel (vel@LaTeXTemplates.com) and
% Arthur Gouinguenet (arthur.gouinguenet@free.fr)
%
% License:
% The MIT License (see included LICENSE file)
%
%%%%%%%%%%%%%%%%%%%%%%%%%%%%%%%%%%%%%%%%%

%----------------------------------------------------------------------------------------
%	PACKAGES AND OTHER DOCUMENT CONFIGURATIONS
%----------------------------------------------------------------------------------------

\documentclass[letterpaper]{twentysecondcv} % a4paper for A4
% \usepackage{mathptmx}

%----------------------------------------------------------------------------------------
%	Color of the theme
%----------------------------------------------------------------------------------------

% DarkTeal
\definecolor{myprimarycolorlight}  {RGB}{86, 106, 106} % For side background 
\definecolor{myprimarycolordark}   {RGB}{79, 75, 65}  % For main text  

% Sand
\definecolor{mysecondarycolordark} {RGB}{203, 182, 156} % For side text
\definecolor{mysecondarycolorlight}{RGB}{250, 250, 250} % For main background

%----------------------------------------------------------------------------------------
%	 PERSONAL INFORMATION
%----------------------------------------------------------------------------------------

% If you don't need one or more of the below, just remove the content leaving the command, e.g. \cvnumberphone{}

\profilepic{profile_zoom.jpg}                                          % Profile picture
\cvname{Arthur Gouinguenet (22)}                                       % Your name
\cvjobtitle{Engineer in High Performance Computing}                    % Job title/career
\cvdate{}                                                              % Date of birth
\cvaddress{Nice, France}                                               % Short address/location, use \newline if more than 1 line is required
\cvnumberphone{+33 6.51.00.69.02}                                      % Phone number
\cvmail{arthur.gouinguenet@free.fr}                                    % Email address
\cvsite{}                                  % Personal website
\cvtwitter{https://x.com/Arthos222}                                    % Twitter
\cvlinkedin{https://www.linkedin.com/in/arthur-gouinguenet-b94119252/} % Linkedin
\cvgithub{https://github.com/ArthurGouing}                             % Github

%----------------------------------------------------------------------------------------

\begin{document}

%----------------------------------------------------------------------------------------
% COMPUTER SKILLS
%----------------------------------------------------------------------------------------

\computerskills{{Python},{C++},{Fortran},
								{OpenGL / Cuda / OpenACC},{MPI / OpenMP},
								{Blender}, {Abaqus}, {Unity}, {Latex / Office Suite}}

%----------------------------------------------------------------------------------------
% LANGUAGE SKILLS
%----------------------------------------------------------------------------------------

\languageskills{{French: native speaker},
							 {English: B2 (Toeic 850)},
							 {Italian: A2},
							 {Good knowledge of LPC (it’s the french cued speech, permitting to communicate with deaf persons)}
							}

%----------------------------------------------------------------------------------------
% INTEREST
%----------------------------------------------------------------------------------------

\interest{{Tricking and gymnastics},
					{Digital sculpting and modelling},
					{Piano}
				 }

%----------------------------------------------------------------------------------------

\makeprofile % Print the sidebar

%----------------------------------------------------------------------------------------
%	 ABOUT ME
%----------------------------------------------------------------------------------------

\aboutme{}
% \aboutme{Engineering student  in computer simulations, with specialization in computer graphics. Proficient in problem-solving and collaborative teamwork. Seeking job opportunities upon graduation in September 2023.}
\makemytitle % Print the title

%----------------------------------------------------------------------------------------
%	 EDUCATION
%----------------------------------------------------------------------------------------

\section{Education}
\\
\begin{twenty} % Environment for a list with descriptions
	%\twentyitem{date}{Title {\normalfont normal text}}{Place}{Description}
	\twentyitem{2022-2023}{Specialisation Semester at Graduate School of Grenoble}{High Performance Computing}
	{Followed computer graphics specialization with a focus on coursework covering HPC, GPU computing, rendering, animation and mathematical optimization. }
	\twentyitem{2020-2023}{M.Sc at Graduate School of Bordeaux}{Mathematics and Mechanics}
	{Learning to develop numerical simulation to solve solid and fluid mechanics real physical problems.}
	\twentyitem{2018-2020}{C.P.G.E at Lycée Masséna}{Physics and Engineering Science}
	{A 2-year intensive curriculum in mathematics and physics, preparing students for entrance examinations to engineering schools in France.}
	\twentyitem{2015-2018}{High school at Lycée Masséna}{Science}{Specializing in mathematics and physics.}
	%\twentyitem{<dates>}{<title>}{<location>}{<description>}
\end{twenty}

%----------------------------------------------------------------------------------------
%	 EXPERIENCE
%----------------------------------------------------------------------------------------


\section{Experience}
\\
\begin{twenty} % Environment for a list with descriptions
	%\twentyitem{date}{Title}{Place}{Description}
	\twentyitem{2024 \\ \emph{(1 year)}}{R\&D Engineer HPC developer}{INRIA Shopia-Antipolis}
	{Porting DIOGENeS, a HPC nanophotonics code, on GPU using OpenACC.
	Performed numerical design optimisation of Meta-surface.}
	% Leading workshop and learning ressource around DIIOGENeS software suite}
	\twentyitem{2023 \\ \emph{(6 months)}}{HPC Engineer Intern}{CERFACS}
	{Porting a fluid mechanics code on GPU using OpenACC and OpenMP. Developing coding tools for benchmarking and code autocompletion.}
	\twentyitem{2022 \\ \emph{(4 months)}}{Engineer Intern}{I2M Bordeaux}
	{Developed a Python script to automate file testing for the non-regression of Notus, a massively parallel code for Computation Fluid Dynamics.}
	%\twentyitem{<dates>}{<title>}{<location>}{<description>}
\end{twenty}

%----------------------------------------------------------------------------------------
%	 PROJET
%----------------------------------------------------------------------------------------

\section{Project}
%--- Personal Project -------------------------------------------------------------------
\\
\subsection{Personal Project}

\begin{twentyshort} % Environment for a short list with no descriptions
	\twentyitemshort{GPU Cloth:}{Blender Add-on (\href{https://x.com/Arthos222}{Blender extension page}) which uses Taichi python library to develope portable GPU code for simulating soft Body in real time, faster than Blender Simulation System.}
	\twentyitemshort{Simuscle:}{Proof of concept product (\href{https://simuscle.vercel.app/}{with website}) linked with Blender add-on to run muscle simulation. Simuscle is developped in C++, using OpenGL and DearImGui.}
	%\twentyitemshort{<dates>}{<title/description>}
	% \twentyitemshort{Peleiz \& Hanisa:}{1 Week-end Game Jam Creation of small game in a team of 8. Mainly working on the graphics and animations. (\href{https://clemetayer.itch.io/peleizhanisa}{link})}
\end{twentyshort}

%--- School Project ---------------------------------------------------------------------

\subsection{School Project}

Developed Various type of Code, Simulator and Application : 
\\
\setlength\tabcolsep{0pt}
\begin{tabular*}{\linewidth}{@{\extracolsep{\fill}} lll }
\textbf{Ray tracer}, & \textbf{Motion capture app}, & \textbf{Rigid body simulator}, \\
\textbf{Pyrolysis simulation}, & \textbf{Finger biomechanics}, & \textbf{Ankle model code}, \\
\textbf{Linear algebra solveur library} & &
\end{tabular*}
\\ \smallskip
Detail information for each project can be found on my \href{\cvgithub}{Github} page. 

\cvnewpage % Start a new page

% \subsection{School Project}
% 
% \begin{twentyshort} % Environment for a short list with no descriptions
% 	\twentyitemshort{Ray Tracer}{Developed Qt applications for raytracing rendering and motion capture animation using OpenGL and C++. }
% 	\twentyitemshort{Motion Capture App}{ Developed Qt application for motion capture 
% animation visualization using OpenGL and C++.}
% 	\twentyitemshort{Rigid Body Simulator}{ Developed real-time physics engine for rigid
% body collision, using convex analysis framework.}
% 	\twentyitemshort{Pyrolysis Simulation}{Simulated the pyrolysis effect of a thermic shield during atmospheric entry. Solved coupled heat equation using adaptative mesh methods in finite volume using C++. }
% 	\twentyitemshort{Finger Biomechanics Simulation}{Biomechanics modelling of climber’s finger. Computing constraints applied on tendons with Abaqus, a mechanical calculation software. }
% 	\twentyitemshort{Ankle Biomechanics Model}{Developed a simulation of ankle articulation during the walk cycle by utilizing existing models and testing them in Fortran90.}
% 	%\twentyitemshort{<dates>}{<title/description>}
% \end{twentyshort}

%----------------------------------------------------------------------------------------
%	 SECOND PAGE EXAMPLE
%----------------------------------------------------------------------------------------


\section{Other information}

\subsection{Review}

Alice approaches Wonderland as an anthropologist, but maintains a strong sense of noblesse oblige that comes with her class status. She has confidence in her social position, education, and the Victorian virtue of good manners. Alice has a feeling of entitlement, particularly when comparing herself to Mabel, whom she declares has a ``poky little house," and no toys. Additionally, she flaunts her limited information base with anyone who will listen and becomes increasingly obsessed with the importance of good manners as she deals with the rude creatures of Wonderland. Alice maintains a superior attitude and behaves with solicitous indulgence toward those she believes are less privileged.

\section{Color Palette}
\drawpalette

%----------------------------------------------------------------------------------------

\end{document} 
